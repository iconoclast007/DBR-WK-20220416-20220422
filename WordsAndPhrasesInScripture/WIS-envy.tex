\subsubsection{The word ``envy"}

Forms of the word ``envy'' found in scripture are: ``envy,'' (19 times)  `envying,'' (5 times) and ``envyings'' (1 time). Job 5:2 tells us that ``envy'' kills,\footnote{\textbf{Job 5:2} - For wrath killeth the foolish man, and envy slayeth the silly one.}  Proverb 23:17 warns us to not envy.\footnote{\textbf{Proverb 23:17} - Let not thine heart envy sinners: but be thou in the fear of the LORD all the day long.} We find that ``envy'' is a powerful enemy in Proverb 27:4.\footnote{\textbf{Proverb 27:4} - Wrath is cruel, and anger is outrageous; but who is able to stand before envy?}. This ``envy,'' though, perished with the one that has it (Ecclesiastes 9:6).\footnote{\textbf{Ecclesiastes 9:6} - Also their love, and their hatred, and their envy, is now perished; neither have they any more a portion for ever in any thing that is done under the sun.}\\
\\
Romans 1:20 identifies envy as one of the sign-posts on the road to depravity.\footnote{\textbf{Romans 1:29} - Being filled with all unrighteousness, fornication, wickedness, covetousness, maliciousness; full of envy, murder, debate, deceit, malignity; whisperers,} and warns against it in Romans 13:13.\footnote{\textbf{Romans 13:13} - Let us walk honestly, as in the day; not in rioting and drunkenness, not in chambering and wantonness, not in strife and envying.} Paul speaks of ``envying'' as a mark of carnality in the church at Corinth in 1 Corinthians 3:3.\footnote{\textbf{1 Corinthians 3:3} - For ye are yet carnal: for whereas there is among you envying, and strife, and divisions, are ye not carnal, and walk as men?} Paul feared that he would find these ``envyings'' at Corinth upon a future visit.\footnote{\textbf{2 Corinthians 12:20} - or I fear, lest, when I come, I shall not find you such as I would, and that I shall be found unto you such as ye would not: lest there be debates, envyings, wraths, strifes, backbitings, whisperings, swellings, tumults:}
\noindent Combinations of words ``envied'', ``envies'', ``envieth,'' and ``envious'' are found 12 times in scripture (KJV). The references document actions taken as a result of this envy, leading to bad results.
\begin{compactenum}
	\item In \textbf{Genesis 26:14}, the Philistines envied Isaac, his success, his substance, his servants, his livestock!\footnote{\textbf{Genesis 26:14} - 
For he had possession of flocks, and possession of herds, and great store of servants: and the Philistines envied him.}
	\item In \textbf{Genesis 30:1} Rachel envied Leah for giving birth.\footnote{\textbf{Genesis 30:1} - And when Rachel saw that she bare Jacob no children, Rachel envied her sister; and said unto Jacob, Give me children, or else I die.}
	\item In \textbf{Genesis 37:11}, Joseph's brethren envy him, leading eventually to betrayal (this is a wonderful type of the Pharisee's envy of Jesus).\footnote{\textbf{Genesis 37:11} - And his brethren envied him; but his father observed the saying.}
	\item In \textbf{Psalm 37:1}, believers are warned to not be envious of evildoers.\footnote{\textbf{Psalm 37:1} - Fret not thyself because of evildoers, neither be thou envious against the workers of iniquity.}
	\item In \textbf{Psalm 73:3}, the psalmist tells of his envy of the foolish and their propserity.\footnote{\textbf{Psalm 73:3} - For I was envious at the foolish, when I saw the prosperity of the wicked.}
	\item \textbf{Psalm 106:16} speaks of the envy of Moses and Aaron.\footnote{\textbf{Psalm 106:16} - They envied Moses also in the camp, and Aaron the saint of the LORD.}
	\item \textbf{Proverb 24:1} warns to not be envious of evil men.\footnote{\textbf{Proverb 24:1} - Be not thou envious against evil men, neither desire to be with them.}
	\item \textbf{Proverb 24:19} says to not fret over evil men (and their apparent success), for the reward is temporary.\footnote{\textbf{Proverb 24:19} - Fret not thyself because of evil men, neither be thou envious at the wicked;}
	\item \textbf{Ecclesiastes 4:4} warns that travail and work often result merely being in a position that is envied by the neighbour.\footnote{\textbf{Ecclesiastes 4:4} - Again, I considered all travail, and every right work, that for this a man is envied of his neighbour. This is also vanity and vexation of spirit.}
	\item In \textbf{Ezekiel 31:9}, the Assyrian as a cedar in Lebanon, with a ``multitude of branches'' was envied by the other trees.\footnote{\textbf{Ezekiel 31:9} - Again, I considered all travail, and every right work, that for this a man is envied of his neighbour. This is also vanity and vexation of spirit.}
	\item \textbf{1 Corinthians 13:4} tells us that charity enveith not.\footnote{\textbf{1 Corinthians 13:4} - Charity suffereth long, and is kind; charity envieth not; charity vaunteth not itself, is not puffed up,}
	\item \textbf{1 Peter 2:1} tells us to lay aside envies.\footnote{\textbf{1 Peter 2:1} - Wherefore laying aside all malice, and all guile, and hypocrisies, and envies, and all evil speakings,}
\end{compactenum}