\subsection{1 Kings 16 Notes from Chris Robbins}
Chapter sixteen gives historical accounts from the prophecy against Baasha to the beginning of the evil reign of Ahab, the son of Omri and husband of Jezebel daughter of Ethbaal king of the Sidonians (a very pagan and evil society to the north of Israel in Phoenicaia). Baasha’s prophecy of destruction over his house comes from Jehu the son of Hanani because of the evil he did in the sight of the LORD walking the ways of Jeroboam and making Israel sin. He provoked the LORD to anger and God would consume his house (1 Kings 16:1-5). This happens shortly (2 years) after Baasha’s son, Elah, becomes king in his place. Zimri, commander of half of Israel’s chariots, conspired against the king and struck him dead in Tirzah while Elah was drunk in the house of Arza, who was over the household at Tirzah (1 Kings 16:6-9). Therefore, Zimri becomes king in his place and fulfills the prophecy by killing every male in the household of Baasha, all his relatives and his friends (1 Kings 16:10-14). Zimri’s act created civil unrest. He lasted only seven days as the king of Israel as Omri rose to power from the battlefield of Gibbethon coming to Tirzah where his supporters besieged the city. With all hope gone, Zimri went into the citadel of the king’s house and burned it up with fire with him in it, thus committing suicide (1 Kings 16:15-20). This sent Israel into another civil division where half followed Omri and the other half followed Tibni the son of Ginath. Omri eventually after 5 years prevailed over Tibni and reigned for a total of 12 years. Half of that time he reigned from Tirzah and half of the time he reigned from Samaria, which he bought from Shemer for two talents of silver. He built Samaria on this hill doing evil in the sight of the LORD acting more wickedly than all who were before him by walking in the ways and example of Jeroboam in all his sins which he made Israel sin, provoking the LORD God of Israel with their idols (1 Kings 16:21-27). Omri was noted in the annuals of Assyrian history and was probably more internationally famous than this account in 1 Kings gives him credit. *Application* Fame in the world is not right standing and fame with God and His perfect revelation in the Word (Bible).\footnote{\href{http://pastorchrisrobbins.blogspot.com/2013/06/1-kings-bible-study-notes-chapter-16.html}{Chris Robbins' notes on 1 Kings 16} }\\
\\
After his death, his son Ahab becomes king and begins a 22 year reign in Samaria that far exceeded any of his ancestors of the Northern Kingdom’s throne in vile evil in the sight of the LORD (1 Kings 16:28-31). He married the pagan Jezebel, who was the most wicked and controlling woman in all of Scripture, and erected an alter for Baal to worship this false god in a temple house he built in Samaria (1 Kings 16:31b-32). He also made the Asherah (a wooden symbol of female deity), which provoked the LORD God of Israel more than all the kings of Israel who were before him (1 Kings 16:33). In the days of Ahab, Hiel the Bethelite built Jericho back up and laid a foundation with the loss of his firstborn, Abiram, and he set up the gates of the city with the loss of his youngest son Segub, according to the word of the LORD spoken by the prophet Joshua son of Nun (Joshua 6:26, 1 Kings 16:34).

\subsection*{Application} Obviously the despicable wickedness of the Northern Kingdom becomes more and more apparent as they defiled anything sacred in the land and kept provoking the LORD to more and more anger with their actions. It was a downward spiral of magnitude proportions of increasing evil and debauchery with a total lack of love or concern for the God who brought them out of Egypt and established them. We too can forget about our gracious and good God when we let the world in and capitulate to its demands for greed, lust, idolatry, and the like. Don’t become consumed with material and expedient gain and forget the LORD who saves you from this present evil age. Though there may be pain in the offering at times as we fight the good fight, victory in the end is assured and the LORD will come through for His faithful ones. Remain steadfast and don’t falter in the battle for righteousness!