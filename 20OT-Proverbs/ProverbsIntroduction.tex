\chapter{Proverbs Introduction}

\section{Proverbs Introduction}
Wisdom has become something of an industry in the United States. Talk radio hosts and syndicated columnists develop devoted followers of advice-seekers. Professional consultants help companies of all sizes solve thorny problems.\footnote{\href{https://tabletalkmagazine.com/article/2007/02/proverbs/}{Robert Rothwell in February 2007 Table Talk}}\marginpar{\scriptsize \centering \fcolorbox{bone}{lime}{\textbf{PROVERBS AND WISDOM}}\\ (Proverbs) 
\begin{compactenum}[I.][8]
    \item \textbf{What is Wisdom Anyway?} \index[scripture]{Proverbs!Pro 30:01}(Pro 30:1)
    \item \textbf{Realizes his Brutish Condition} \index[scripture]{Proverbs!Pro 30:02-03}(Pro 30:2-3)
    \item \textbf{Respects God's Word} \index[scripture]{Proverbs!Pro 30:04-06}(Pro 30:4-6)
    \item \textbf{Requests Security from God} \index[scripture]{Proverbs!Pro 30:07-09}(Pro 30:7-9)
    \item Is \textbf{Repulsed by Wickedness} \index[scripture]{Proverbs!Pro 30:10-14}(Pro 30:10-14)
    \item \textbf{Recognizes Truth in Nature} \index[scripture]{Proverbs!Pro 30:15-31}(Pro 30:15-31)
    \item \textbf{Repents of Self-Promotion} \index[scripture]{Proverbs!Pro 30:32}(Pro 30:32)
\end{compactenum} }

Humanity’s long quest for the wisdom of the ages continues today. As Christians we know that wisdom is a gift from God, found primarily in the pages of sacred Scripture. In the Old Testament, the Proverbs of Solomon stand out as the place to find wisdom, and so it will profit us to look at how we can properly understand and apply this book’s teaching.\\
\\
What Is Wisdom?\\
\\
As the Holy Spirit inspired Proverbs to help us attain wisdom (1:2), understanding this book requires us to explore the nature of wisdom. Simply put, wisdom is “skill,” or “expertise.” Wise people live life well; they avoid common problems and handle other ones with insight. Like many small animals, wise men and women master their domains in spite of their limitations (30:24–28).

According to Proverbs, wisdom is rooted in the “fear of the Lord” (1:7), which characterizes those who obey His law (Ps. 34:11–16; Acts 5:29). The fear of the Lord has an intellectual component: we must study and memorize God’s commandments to know and follow His will (Deut. 6:4–9). But the fear of the Lord is also an emotional response of love for the Father and trusting obedience to His commands (Mark 10:28–31; James 2:14–26; 1 John 4:16). Satan can quote Scripture, but He does not love the Lord and therefore foolishly rebels against Him (Matt. 4:1–11). Jesus calls the rich man a “fool” because he had no regard for his Creator — not because his life lacked wisdom altogether (Luke 12:13–21). 

Wisdom is a virtual synonym for righteousness in the book of Proverbs — the prologue tells us these proverbs are given for wisdom and righteousness (1:3). Wise teaching and righteous living produce life (12:28; 13:14), but the godless person and the fool wander the wide road leading to death (10:14; 11:7). Clearly, we cannot be wise without holiness, and we cannot be holy if we do not seek after wisdom (see also Matt. 6:33).

Proverbs complements the other biblical books by reminding us that common, every-day life is an occasion for great service to our Creator. Most of us will not wield geopolitical influence or direct the course of the church. Nevertheless, the Lord cares deeply about our lives and keeps a careful eye on all our actions (Prov. 5:21). Proverbs reminds us of this awesome reality and gives us tangible ways we can obey God’s law. For example, if we rejoice in the wife (or husband) of our youth (vv. 15–20), we will look for ways to celebrate the emotional and sexual relationship with our spouses and thus be less inclined to violate our marriage vows. 

Such passages remind us the Lord sanctifies relationships between “ordinary” people. We are not “lone ranger Christians,” we must live life in community with other believers. Fulfilling Proverbs’ many exhortations to confess sins (for instance, 28:13) means we are real with God and with others. Wise people seek Christians to whom they can be accountable for righteousness. They look for churches where sins are healthily acknowledged and where believers bear one another’s burdens (Gal. 6:2). People who make decisions without listening to godly friends are fools (Prov. 15:22). Western individualism tells us to make choices on our own. Proverbs teaches us that we do not live private lives; only simpletons do not heed the time-honored wisdom found in the community of God’s people (1:8; 4:1–6; 24:6). \\
\\
How to Read Proverbs\\
\\
A prayerful reading of this book is key to becoming wise (James 1:5). But like other literature, we must pay attention to the genre and setting of Proverbs to ensure its proper interpretation. Lest we misappropriate these wise sayings, let us remember four principles:

A single proverb is not designed for every circumstance in this life. We do not expect an uninspired proverb to apply at all times. The same maxim applies to the Spirit-inspired Proverbs of Solomon. Dr. R.C. Sproul uses the English proverbs “look before you leap” and “he who hesitates is lost” to illustrate this point. There are occasions when we need to tread carefully before making a decision — choosing a spouse comes to mind. However, hesitation is foolish at other times. For example, we never pause to consider whether we should stop our two-year old from crossing the highway by himself. Likewise, if we expect one of Solomon’s proverbs to be true in every instance, we will be disappointed and confused. Whether or not we should answer a fool according to his folly (Prov. 26:4–5) depends on the person with whom we are dealing. 

Research the problem that presents itself thoroughly. Numbers 35:9–28 did not mandate capital punishment for every killing; rather, it was only for premeditated murder. In order to determine the proper punishment, the authorities had to investigate whether the crime was planned. Rightly using God’s proverbs and laws requires knowledge of the circumstances to which they must apply.

When reading one proverb, keep all of them in mind. Context matters — correctly interpreting one proverb only happens when we consider it in light of the others. All of the proverbs must be ready on our lips (Prov. 22:17–18). “Train up a child in the way he should go; even when he is old he will not depart from it” (v. 6) tells us that godly parents usually raise godly children. But other assumptions of Proverbs have to be met if the child is to remain on the straight and narrow. Children must heed the godly wisdom of their parents and elders and have hearts inclined toward God if they are to remain faithful (1:8–9, 32–33; 3:5–6; 7:1–3). If we ignore the other proverbs, we may illegitimately cling to “train up a child” and assume that raising children in a thoughtful and deliberate Christian home necessarily means those children will become believers. Remembering the proverb’s context moves us to disciple those raised in the faith even when they are older, because we know teaching heard long ago profits nothing if it is abandoned today. Moreover, when we read “train up a child” in light of all the other proverbs, we will not use it to automatically condemn the parenting skills of those with ungodly offspring. All of Proverbs, as well as the entire Bible, shows us that faithful parents sometimes produce faithless children. Even fathers and mothers who diligently teach God’s Word to their little ones (Deut. 6:4–9) cannot exchange a heart of stone for heart of flesh.

Keep the end in view. Many proverbs predict success for the Lord’s people, and, indeed, those who live righteous lives usually elude difficulty and live at peace with others (Prov. 12:21; 16:7). Yet while holy men and women often find “riches and honor and life” (22:4), we all know faithful servants who suffer. Proverbs recognizes this reality as well. It is possible to fear God and yet live in poverty (15:16; 19:1). There will be times when wickedness brings earthly treasure (10:2a). If we forget these truths and look at the proverbs offering success for the righteous as absolute promises, we will be discouraged when experience does not match reality. We might also become like Job’s friends who erroneously thought his troubles proved that he was guilty of sin.

However, the fact that proverbs are not automatic promises for this present life does not mean there is no guarantee of final success for the righteous. Scripture’s witness to the Lord’s justice (Gen. 18:25; Rev. 16:5) points to a time in which God’s people are vindicated and the wicked are destroyed. For God to uphold justice, He must right the wrongs done to His holy ones in a life outlasting the grave. This hope is shadowy in Proverbs (see 10:2b, 25; 11:21; 16:4), more a necessary consequence than a direct teaching. Nevertheless, the proverbs looking to great blessing for the righteous will be true in an ultimate sense, and we therefore look forward to that day (Dan. 12:1–3; Rev. 20:11–15).\\
\\
 Proverbs and Christ\\
 \\
In pointing toward an afterlife, Proverbs anticipates the One who will vindicate the righteous and reward them for their service. If steadfast love and righteousness preserve the king (Prov. 20:28), only a ruler who perfectly embodies these qualities can qualify as the vindicator of the holy. This Messiah is the Lord Jesus Christ, who not only submitted perfectly to the wisdom of Proverbs, He is also the very wisdom of God (1 Cor. 1:24). Solomon would die a fool (1 Kings 11), but Jesus always feared God and shunned evil (Prov. 3:7; 1 Peter 2:22). If we read Proverbs through the fuller revelation of His teaching and submit to its precepts, we will wisely live to the glory of God.  

\newpage
\subsection{Proverbs and Wisdom}

\subsubsection{What is Wisdom?}
\begin{compactenum}
    \item Wisdom is dispensed by God. \\
    \\
    In 1  Kings 3:9 Solomon asks : Give therefore thy servant an understanding heart to judge thy people, that I may discern between good and bad: for who is able to judge this thy so great a people? 3:12 says Behold, I have done according to thy words: lo, I have given thee a wise and an understanding heart; so that there was none like thee before thee, neither after thee shall any arise like unto thee. Solomon would later abandon this wisdom.\\
    \\
    James 1:5 tells us If any of you lack wisdom, let him ask of God, that giveth to all men liberally, and upbraideth not; and it shall be given him. We are to ask God for the wisdom we need.  Eph 1:17 says ``That the God of our Lord Jesus Christ, the Father of glory, may give unto you the spirit of wisdom and revelation in the knowledge of him:'' Col 1:9 says ``For this cause we also, since the day we heard it, do not cease to pray for you, and to desire that ye might be filled with the knowledge of his will in all wisdom and spiritual understanding;'' \\
    \\
    But for those who are not interested in this wisdom, God does not force it on them.
    \item Wisdom is close to righteousness and it has much practical value!\\
    \\
    Proverb 1:2 tells us that the purpose of the proverbs is ``to know wisdom.'' Wisdom is then defined in Proverb 1:7 as ''the fear of the LORD.'' We could say that wisdom is ``skill'' or ``expertise'' to use knowledge to live correctly.  The fear of the LORD  means that there is a ``correct'' way. Wisdom helps us to avoid common problems and mistakes, to avoid trouble. Practically speaking, those who follow the Bible to live are just less susceptible to trouble because they avoid it.  Proverbs has much to add to this thing called ``wisdom.'' Proverb 1:8 tells us, further, that there is a group of people called ``fools'' who despise and avoid wisdom.\\
    \item Wisdom, as taught in Proverbs, has much to do with living with others.\\
    \\
    There is this thing called community, the place where we live and the people who live there. Western society tells us to be self-sufficient and an individual.  Proverbs, and scripture, tells us to live right with mankind. Proverb 15:22 says ``Without counsel purposes are disappointed: but in the multitude of counsellors they are established.'' So, Proverbs tells us that wise people seek Christians for godly counsel and for accountability. A politician said that ``it takes a village to raise a child.'' It does, just not the village she was referring to. So, wisdom sets the context for bodies of believers, in the New Testament called churches\\
    \item Many Proverb are repeated and repeated.\\
    \\
    Many of the thing in proverbs are said over and over again. On thing, for example, are all  the warnings against immoral women.  Equally, proverbs speaking of the trap of intoxication are scattered throughout Proverbs. Proverbs about proper speech are spread through the book. This is because, I, we, need to be reminded over and over again.\\
    \item There is no single proverb  designed for every circumstance in this life.\\
    \\
    They are inspired, but are situational. For example consider our English proverbs “look before you leap” and “he who hesitates is lost” to illustrate this point. There are occasions when we need to tread carefully before making a decision — choosing a spouse comes to mind. However, hesitation is foolish at other times. For example, we never pause to consider whether we should stop our two-year old from crossing the highway by himself. Consider Proverbs 26:4-5 which say ``Answer not  a fool accordingly to his folly, lest thou also be like unto him'' and ``Answer a foll according to his folly, lest he be wise in his eyes.
    \item proverbs are not automatic promises for this present life does not mean there is no guarantee of final success for the righteous.\\
    \\
    We know of plenty of examples where godly people suffered unjustly or were hit with incurable maladies. Scripture’s witness to the Lord’s justice (Gen. 18:25; Rev. 16:5) points to a time in which God’s people are vindicated and the wicked are destroyed. For God to uphold justice, He must right the wrongs done to His holy ones in a life outlasting the grave.
    \item How does one get wisdom?\\
    \\
    By observation\\
    By experience\\
    By Pain\\
    By cost\\
    By listening and learning.
    
\end{compactenum}

